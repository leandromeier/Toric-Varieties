\documentclass[a4paper]{article}

\title{Summary of the T-varieties seminar}
\author{Leandro Meier}
\date{}

\usepackage{amsmath}
\usepackage{amsfonts}
\usepackage{amssymb}
\usepackage{amsthm}

\usepackage{enumerate}
\usepackage{booktabs}
\usepackage{graphicx}
\usepackage[hidelinks]{hyperref}

\usepackage{comment}
\usepackage{bm}

\DeclareMathOperator{\TV}{TV}
\DeclareMathOperator{\tail}{tail}

\setlength\parindent{0pt}

\newtheorem{theorem}{Theorem}[section]
\newtheorem{proposition}[theorem]{Proposition}
\newtheorem{lemma}[theorem]{Lemma}

\theoremstyle{definition}
\newtheorem{definition}[theorem]{Definition}

\begin{document}
\maketitle
\tableofcontents
\section{Toric Varieties}
\subsection{Affine Toric Varieties}
\begin{definition}
  A \emph{toric variety} is an irreducible variety (in this, usually affine or projective) $V$ such that 
  \begin{enumerate}[(i)]
    \item $\left( \mathbb{C}^{*} \right)^{n}$ is an open subset of $V$ and
    \item the action of $\left( C^{*} \right)^{n}$ on itself extens to an action of $\left( \mathbb{C}^{*} \right) ^{n}$ on $V$.
  \end{enumerate}
\end{definition}
  Examples are $\mathbb{C}^{n}$, $\left( \mathbb{C}^{*} \right) ^{n}$ and $\mathbb{P}^{n}$, for the latter, use that $\left( \mathbb{C}^{*} \right) ^{n}$ can be identified with one of the open subsets $U_{i} = \left\{\left( a_{0} \colon \dots \colon a_{n+1} \right) \mid a_{i} \neq 0\right\}$.

  Let $\boldsymbol{a} \in \mathbb{Z}^{n}$.\\
  The map $\left( \mathbb{C}^{*} \right) ^{n} \rightarrow \mathbb{C}^{*}$ given by $\boldsymbol{t} \mapsto \boldsymbol{t^{a}}$ is called a \emph{character}.\\
  A \emph{1-parameter subgroup} $\boldsymbol{\lambda^{a}} \colon \mathbb{C}^{*} \rightarrow \left( \mathbb{C}^{*} \right) ^{n}$ is given by $\boldsymbol{\lambda^{a}} \left( t \right) = \left( t^{a_{1}}, \dots, t^{a_{n}} \right) $.

  \begin{definition}
    A \emph{rational polyhedral cone} $\sigma \subseteq \mathbb{R}^{n}$ is something of the form
    \[
      \sigma = \left\{ \lambda_{1}\boldsymbol{u} _{1} + \dots + \lambda_{n}\boldsymbol{u} _{l} \in \mathbb{R}^{n}\mid \lambda_{1}, \dots, \lambda_{l} \geq 0\right\},
    \]
    for $\boldsymbol{u} _{1}, \dots, \boldsymbol{u} _{l} \in \mathbb{Z}^{n}$. We further define:
    \begin{itemize}
      \item $\sigma $ is \emph{strongly convex} if $\sigma \cap - \sigma = \left\{ 0\right\} $.
      \item The dimension of $\sigma$ is the dimension of the smallest subspace of $\mathbb{R}^{n}$ that contains $\sigma$.
      \item A \emph{face} of $\sigma$ is the intersection of the set $\left\{\ell = 0\right\}$ with $\sigma$ where $\ell$ is a linear form that is nononnegative on $\sigma$.
      \item 1-dimensional faces are called \emph{edges}. For an edge $\rho$, its  \emph{primitive element} is the unique generator $\boldsymbol{n} _{\rho}$ of $\rho \cap \mathbb{Z}^{n}$. The cone is generated by the $\boldsymbol{n} _{\rho}$ for all its edges $\rho$.
      \item Codimension-1 faces are called \emph{facets.} 
    \end{itemize}
  \end{definition}
  

  \begin{definition}
    For a cone (strongly convex, rational, polyhedral) $\sigma \subseteq \mathbb{N}_{\mathbb{R}}$, define its dual cone 
    \[
      \sigma ^{ \vee} = \left\{ \boldsymbol{m} \in M_{\mathbb{R}}\mid \left\langle \boldsymbol{m} ,\boldsymbol{u}  \right\rangle\geq 0 \text{ for all } \boldsymbol{ u } \in \sigma\right\}.
    \]
    Here $\left\langle \cdot, \cdot \right\rangle$ denotes the natural pairing between $N$ and its dual lattice $M$. In the case of $N \cong M \cong \mathbb{Z}^{n}$, this is just the usual scalar product. 
  \end{definition}

  Using the dual cone $\sigma^{\vee}$, we can construct a variety $U_{\sigma}$ as follows. Consider the lattice points of $\sigma^{\vee}$: $\sigma^{\vee} \cap \mathbb{Z}^{n}$. The lattice points are finitely generated (Gordan's Lemma). Let $\boldsymbol{m} _{1}, \dots, \boldsymbol{m} _{l}$ be generators, and consider the map 
  \[
    \varphi \colon \left( \mathbb{C}^{*} \right) ^{n} \rightarrow  \mathbb{C}^{l}
  \]
  defined by $\varphi \left( \boldsymbol{t}  \right) = \boldsymbol{t} ^{m_{1}} \dots, \boldsymbol{t} ^{m_{l}}$ and let $U_{\sigma}$ be the Zariski closure of $\varphi$.\\
  One can prove that this is a toric variety, and that the $\boldsymbol{t^{m}} $ are defined everywhere on $U_{\sigma}$.\\
  The coordinate ring of $U_{\sigma}$ is given by $\mathbb{C} \left[ \sigma^{\vee} \cap \mathbb{Z}^{n} \right]$, which is a notation for the ring of Laurent polynomials over $\mathbb{C}$ generated by the $\boldsymbol{t^{m}} $ for $\boldsymbol{m}  \in \sigma^{\vee }\cap\mathbb{Z}^{n}$.

  \begin{theorem}[Theorem 7.2]
    The Zariski closure of the image of the map $\varphi $ from above is the normal afine toric variety $U_{\sigma}$ determined by $\sigma $ and $\mathbb{Z}^{n}$ if and only if $\sigma^{\vee} \cap \mathbb{Z}^{n}$ is generated over $\mathbb{Z}_{\geq 0}$ by $\boldsymbol{m} _{1}, \dots, \boldsymbol{m} _{l}$.
  \end{theorem}

  Alternative construction of $U_{\sigma}$: the Spectrum of the semigroup algebra $\mathbb{C} \left[ \sigma^{\vee} \cap M \right]$. We also refer to the affine toric variety thus obtained by $V_{\sigma}$.

  \subsection{The Toric Variety of a Fan}
  \begin{definition}
    A \emph{fan} is a finite collection $\Sigma$ of cones in $\mathbb{R}^{n}$ such that :
    \begin{itemize}
      \item Each $\sigma \in \Sigma$ is a strongly convex rational polyhedral cone.
      \item If $\sigma \in \Sigma$ and $\tau$ is a face of $\sigma$, then $\tau \in \Sigma$.
      \item If $\sigma, \tau \in \Sigma$, then $\sigma \cap \tau$ is a face of each cone.
    \end{itemize}
  \end{definition}
  Let $\sigma$, $\tau$ be cones in a fan $\Sigma$ and $\tau$ a face og $\sigma$. This implies that $\sigma ^{\vee}  \subseteq \tau ^{\vee} $ and thus also $\mathbb{C} \left[ s_{\sigma} \right] \subseteq \mathbb{C} \left[ S_{\tau} \right]$, hence we get an embedding of $V_{\tau}$ as an open subset of $V_{\sigma}$. More precisely, $V_{\tau}$ is naturally isomorphic to the open subset defined by $\chi^{m}$inside $V_{\sigma}$, since $\mathbb{C} \left[ S_{\tau} \right] = \mathbb{C} \left[ S_{\sigma} \right]_{\chi^{m}}$, where $m \in M$ is such that $H_{m} \cap \sigma = \tau$.\\
  The fact that any two cones in $\Sigma$ intersect in a common face yields immmersions
  \begin{align*}
    V_{\sigma}\cap V_{\sigma'} \rightarrow  V_{\sigma}  & \text{ and } V_{\sigma}\cap V_{\sigma'} \rightarrow  V_{\sigma'}.
  \end{align*}
  Denoting the images of these immersions by $V_{\sigma \sigma'}$ and $V_{\sigma' \sigma}$, respectively, there are isomorphisms between them. Now we have all the data we need to glue an abstract variety, according to the next definition.
  \begin{definition}
    Given  a fan $\Sigma$ in $N_{\mathbb{R}}$, $X_{\Sigma}$ is the abstract toric variety constructed using the gluing data $\left\{ \left\{V_{\sigma}\right\}_{\sigma \in \Sigma}, \left\{V_{\sigma \sigma'}\right\}_{\sigma, \sigma' \in \Sigma}, \left\{g_{\sigma \sigma'}\right\}_{\sigma, \sigma' \in \Sigma}\right\}$.
  \end{definition}
  \begin{theorem}
    The construction above yields a is a normal separated toric variety $X_{\Sigma}$.
  \end{theorem}
  In fact, the torus $T \left( N \right) = N \otimes_{\mathbb{Z}} C^{*} \cong \left( C^{*} \right)^{n}$ is the open subset obtained from the cone $\left\{ 0\right\}$.
  \begin{definition}
    A variety $Y$ is called
    \begin{itemize}
      \item \emph{normal}, if each local ring $\mathcal{O} _{Y,p}$ is normal, i.e. integrally closed in its field of fractions. (should we assume $Y$ irreducible here? yes, but toric varieties are irreducible)
      \item \emph{separated}, if the diagonal map $\Delta \colon Y \rightarrow  Y \times Y$ is closed. (With respect to which topology on $Y \times Y$?)
    \end{itemize}
  \end{definition}

  Some examples like $\mathbb{P}^{2}$, to be added.\\

  %TODO: add orbit-cone-limit correspondence and finite quotient singularities
  
  
  \pagebreak
  \section{Divisors}

  \section{Affine T-varieties}
  We use different notation for the concepts known from the very first part: $M$, $ N$ are still mutually dual lattices. A cone in $N_{\mathbb{Q}}$ spanned by elements $a_{1}, \dots, a_{m}$ is now denoted by $\sigma = \left\langle a_{1}, \dots, a_{m} \right\rangle$ and $\sigma ^{\vee}  $ lives inside $M_{\mathbb{Q}}$. We assume that the $a_{i}$ are primitive elements of $N_{\mathbb{Q}}$, and we denote the affine troic variety obtained from $\sigma$ by $\TV \left( \sigma \right)$.

  \subsection{Toric Bouquets}
  Let $\Delta$ be a polyhedron, i.e. a finite intersection of hyperplanes, in $N_{\mathbb{Q}}$. We define its \emph{tail cone} 
  \[
    \tail \left( \Delta \right) = \left\{a \in N_{\mathbb{Q}} \mid a + \Delta \subseteq \Delta\right\}.
  \]
  Assume now that $\Delta$ is pointed, (i.e. strongly convex), then we can write it as the Minkowski sum  $\Delta = \Delta ^{c} + \tail \left( \Delta \right)$, where $\Delta ^{c}$ denotes the convex hull of the vetices of $\Delta$, which in this case exist.
  
  
  


\end{document}
